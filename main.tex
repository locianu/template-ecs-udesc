% template para o relatório de Estágio Curricular Supervisionado (ECS) da Universidade do
% Estado de Santa Catarina (UDESC), feita por Luciano O. Junior, acadêmico do curso de 
% graduação de Licenciatura em Física.
% esta template segue o modelo segundo o Manual Para a Elaboração de Trabalhos Acadêmicos da UDESC
% 10ª ed.

% configuração do documento
\documentclass[
	12pt,				% tamanho da fonte
	openright,			% capítulos começam em pág ímpar (insere página vazia caso preciso)
	oneside,			% para impressão em recto e verso (twoside). Oposto a (oneside)
	a4paper,			% tamanho do papel. 
	chapter=TITLE,		% títulos de capítulos convertidos em letras maiúsculas
	section=TITLE,		% títulos de seções convertidos em letras maiúsculas
	sumario=tradicional,
	english,			% idioma adicional para hifenização
	brazil,				% o último idioma é o principal do documento
	%fleqn,				% equações alinhadas a esquerda (UDESC/CCT)+
    onehalfspacing,     % espaçamento de linha 1.5
	]{abntex2}
\usepackage[utf8]{inputenc}
\usepackage[T1]{fontenc}
\usepackage[brazil]{babel}
\usepackage[a4paper, left=3cm, right=2cm, top=3cm, bottom=2cm]{geometry}


% formatação geral
\usepackage{titlesec}
\usepackage{enumitem} % permite fazer listas
\usepackage{indentfirst} % (começa o parágrafo com uma identação
\usepackage{float} % imagens flutuando no texto
\usepackage{hyperref} % para linkar elementos (citações, refs, urls, etc)
\usepackage{setspace} % espaçamento de linha
\usepackage{fancyhdr}
\usepackage{setspace}
\usepackage{ragged2e}


% imagens e tabelas
\usepackage{array}
\usepackage{multirow}
\usepackage{booktabs}
\usepackage{booktabs}
\usepackage{graphicx}
\usepackage{tablefootnote}
\usepackage{caption}
\usepackage{xcolor}
\usepackage{tcolorbox} 
\usepackage{tocloft}
\usepackage{newfloat}
\usepackage[acronym, nomain, symbols, section=section, nonumberlist]{glossaries} \makeglossaries
\usepackage[symbols]{glossaries-extra}
% texto
\usepackage{mathptmx} % Times New Roman para texto e equações matemáticas
\usepackage[scaled=.92]{helvet} % Helvetica como fonte sem serifa, em escala 92% de para não ficar feio com a Times New Roman como sugere a documentação
\usepackage{courier} % Courier para monoespaçamentos
\usepackage{lipsum} % texto placeholder

% miscelânea e outras ferramentas
\usepackage{tikz} % para desenhar formas

% para formatar elementos textuais específicos
\usepackage{etoolbox} 
\usepackage{xpatch}

% forçando o uso de Times no documento todo
% \renewcommand{\familydefault}{\rmdefault} % Force roman font
% \AtBeginDocument{\fontfamily{ptv}\selectfont} 

% forçando o uso de Helvetica no documento todo, uma fonte sem serifa para substituir Arial
\renewcommand{\familydefault}{\rmdefault}
\AtBeginDocument{\fontfamily{ptm}\selectfont} 

% usar o etoolbox para reformular comandos
\makeatletter
\appto\TPTnoteSettings{\footnotesize} % 10pt for table footnotes
\makeatother

% legendas
\DeclareCaptionFont{legendas}{\fontsize{10pt}{12pt}\selectfont}
\captionsetup{font=legendas}

% paginação
\fancypagestyle{plain}{% para folha de rosto não numerada
    \fancyhf{} % limpa os cabeçalhos e rodapés
    \renewcommand{\headrulewidth}{0pt}
}
\fancypagestyle{main}{% para páginas com conteúdo textual
    \fancyhf{}
    \fancyhead[R]{\fontsize{10pt}{12pt}\selectfont\thepage} % número de página justificado a direita
    \renewcommand{\headrulewidth}{0pt}
}

% começa a contagem de páginas depois da folha de rosto
\makeatletter
\newcommand{\ignorefortoc}{%
  \addtocontents{toc}{\protect\setcounter{tocdepth}{-10}}
}
\newcommand{\restorefortoc}{%
  \addtocontents{toc}{\protect\setcounter{tocdepth}{5}}
  \setcounter{tocdepth}{5}%
}
\makeatother

\makeatletter
\pretocmd{\tableofcontents}{%
  \cleardoublepage
  \pagenumbering{arabic}% Start counting here
  \setcounter{page}{1}% Reset to page 1
  \setcounter{chapter}{0}% Reset chapter counter
}{}{}
\makeatother

% configurando a lista de figuras/quadros/fotografias/etc
\newcommand{\setupabntlists}{
  \newcommand{\cftfigpresnum}{Figura~}
  \newcommand{\cfttabpresnum}{Tabela~}
  \newcommand{\cftfigaftersnum}{---}
  \newcommand{\cfttabaftersnum}{---}
  \setlength{\cftfigindent}{0pt}
  \setlength{\cfttabindent}{0pt}
  \newcommand{\cftfigleader}{\cftdotfill{\cftdotsep}}
  \newcommand{\cfttableader}{\cftdotfill{\cftdotsep}}
}

\providecommand{\listacronymname}{LISTA DE ABREVIATURAS E SIGLAS}
\providecommand{\listsymbolname}{LISTA DE SÍMBOLOS}

\loadglsentries{PreTextuais/13-1-siglas}
\loadglsentries{PreTextuais/14-1-simb}

\titleformat{\chapter}
  {\normalfont\bfseries\MakeUppercase}
  {\thechapter \quad}
  {0pt}
  {\fontsize{12pt}{14.4pt}\selectfont} 

\titlespacing*{\chapter}{0pt}{30pt}{20pt} 

\titleformat{\section}
  {\normalfont\bfseries\MakeUppercase}
  {\thesection \quad}
  {1em}
  {\fontsize{12pt}{14.4pt}\selectfont}

\titleformat{\subsection}
  {\normalfont\bfseries}
  {\thesubsection \quad}
  {1em}
  {\fontsize{12pt}{14.4pt}\selectfont}

\renewcommand{\sfdefault}{\rmdefault} 
\begin{document}

\ignorefortoc
% primeira folha, capa do trabalho
\thispagestyle{empty}
% identificação da instituição, mude o seu centro conforme necessário
\noindent
\begin{minipage}[t]{0.15\textwidth} % Reduced from 0.25 to fix overflow
  \vspace{0pt} % Top alignment anchor
  \includegraphics[
    height=3\baselineskip, 
    keepaspectratio=true 
  ]{Imagens/Logo_UDESC/cor_vertical_pdf.pdf}
\end{minipage}
\hspace{0.02\textwidth} 
\begin{minipage}[t]{0.90\textwidth}
  \vspace{0.3\baselineskip} % Fine-tuned vertical alignment
  \textbf{UNIVERSIDADE DO ESTADO DE SANTA CATARINA — UDESC} \\
  \textbf{CENTRO DE CIÊNCIAS TECNOLÓGICAS — CCT} \\
  \textbf{DEPARTAMENTO DE FÍSICA — DFIS}
\end{minipage}
\vspace{4cm}

% seu nome aqui
\begin{center}
    \textbf{\large{ACADÊMICO(A)}}
\end{center}

\vspace{2cm}


% criando um retângulo com texto escrito por questões estéticas
\definecolor{myorange}{RGB}{243,186,83}

\begin{flushright}
\colorbox{myorange}{
    \begin{minipage}[t]{9.20cm} 
        \color{white} 
        \vspace{6pt} 
        \hspace*{10pt}\parbox{\linewidth}{
        ESTÁGIO CURRICULAR SUPERVISIONADO IV\\
        \\
        \fontsize{16}{12}\selectfont \textbf{RELATÓRIO FINAL}
        }
        \vspace{6pt} 
    \end{minipage}
}
\end{flushright}
\vfill

\begin{center}
\textbf{JOINVILLE\\
        2025}
\end{center}
\pagebreak

% folha de rosto
\noindent
\begin{center}
    \centering
    \textbf{ACADÊMICO(A)}
\end{center}
\vspace*{\fill}
\begin{center}
    \textbf{RELATÓRIO FINAL DE ESTÁGIO CURRICULAR SUPERVISIONAL IV}
\vspace{3cm}
\end{center}
\par
\hangindent=8cm
\hangafter=0
\noindent{Relatório de Estágio Curricular Supervisionado do curso de Licenciatura em Física, da Universidade do Estado de Santa Catarina, como requisito para cumprimento de horas de estágio.
Orientador: (Nome completo)}
\par
\vfill
\begin{center}
    \textbf{JOINVILLE\\
            2025}    
\end{center}
\pagebreak


% folha de apresentação da escola e supervisor
\noindent
\begin{center}
    \centering    
    PROFESSOR ORIENTADOR DE ESTÁGIO: (Nome completo)\\
    \vspace*{\fill}
    PROFESSOR SUPERVISOR: (Nome completo)\\
    \vfill
    ESCOLA (Nome da escola por extenso)
\end{center}
\pagebreak

% folha de aprovação
\noindent
\begin{center}
    \centering
    \textbf{NOME COMPLETO}\\
    \vspace*{\fill}
    \textbf{ESTÁGIO CURRICULAR SUPERVISIONADO IV:}\\
    RELATÓRIO FINAL\\
    \vfill
\end{center}

Este relatório foi julgado adequado e aprovado em sua forma final.

\vfill
\textbf{Banca Examinadora}
\vfill

\begin{minipage}{\textwidth}
\centering
\textbf{\rule{10cm}{0.01cm}}\\
\textbf{Professor (Nome do professor)} \\
Orientador do Estágio \\[1.5em]
\vspace{1cm}
\textbf{\rule{10cm}{0.01cm}}\\
\textbf{Professor nome completo} \\
Supervisor do Estágio \\[1.5em]
\vspace{1cm}
\textbf{\rule{10cm}{0.01cm}}\\
\textbf{Professor nome completo} \\
Membro Convidado
\end{minipage}
\vfill
\centering
\textbf{JOINVILLE, DD/MM/AAAA}
\pagebreak


% folhas a serem substituídas, importe o pdf e insira no arquivo
\begin{center}
\vspace*{\fill}
\textbf{ADICIONAR AQUI: AVALIAÇÃO FINAL}\\
\vfill
\end{center}
\pagebreak

\begin{center}
\vspace*{\fill}
\textbf{ADICIONAR AQUI: AVALIAÇÃO DO ESTAGIÁRIO PELA ESCOLA}\\
\vfill
\end{center}
\pagebreak
\begin{center}
\vspace*{\fill}
\textbf{ADICIONAR AQUI: TERMO DE COMPROMISSO DE ESTÁGIO CURRICULAR OBRIGATÓRIO}\\
\vfill
\end{center}
\pagebreak
\begin{center}
\vspace*{\fill}
\textbf{ADICIONAR AQUI: PLANO DE ESTÁGIO}\\
\vfill
\end{center}
\pagebreak
\begin{center}
\vspace*{\fill}
\textbf{ADICIONAR AQUI: CRONOGRAMA FÍSICO REAL}\\
\vfill
\end{center}


\pagebreak




% dedicatória
\vfill
\noindent
\vspace*{\fill}
\hspace*{8cm}
\begin{minipage}{\dimexpr\textwidth-8cm} 
    \lipsum*[1][1-2] % substitua isto pela sua dedicatória
\end{minipage}
\vspace{2\baselineskip}


\pagebreak


% agradecimentos
\begin{center}
  \textbf{AGRADECIMENTOS} 
\end{center}

\justifying
\vspace{1.5pt}
\lipsum[10][1-20]
\pagebreak


% epígrafe
\vspace*{\fill}

% cuidado: epígrafes com quatro ou mais linhas utilizam um tamanho de fonte diferente, cheque o manual e atualize sua epígrafe de acordo

% caso sua epígrafe contenha no máximo três linhas, substitua o seguinte bloco pelo seu texto 
% em caso de citação direta:
\noindent
\hspace*{1.25cm}
\begin{minipage}{\dimexpr\textwidth-1.25cm} 
    "\lipsum*[10][1-4]" (Lipsum, 2025, p.420)
\end{minipage}
\vspace{2\baselineskip}


% caso sua epígrafe contenha mais de três linhas, descomente as seguintes linhas: 

% \hspace*{8cm}
% \begin{minipage}{\dimexpr\textwidth-8cm}
  % \lipsum[10][1-8](Lipsum, 2025, p.69)
% \end{minipage}
% \vspace{2\baselineskip}


\pagebreak


% resumo língua vernácula
% Elemento obrigatório que contém a apresentação concisa dos pontos
% relevantes do trabalho, fornecendo visão rápida e clara do conteúdo e das conclusões
% do mesmo. Deve ser elaborado conforme os requisitos estipulados pela NBR 6028
% (ABNT, 2021).
% Deverá conter, de forma clara e sintética, a natureza do trabalho, o objetivo, o
% método, os resultados e as conclusões, visando fornecer elementos para o leitor
% decidir sobre a consulta do trabalho no todo. O resumo deve, conforme ABNT (2021):
% a) ser iniciado com frase significativa, explicando o tema principal do
% documento;
% b) ser redigido em parágrafo único em linguagem clara e objetiva, sem recuo
% de parágrafo na primeira linha;
% c) convém o uso dos verbos deve ser na voz ativa, utilizando-se de preferência
% a terceira pessoa do singular;
% 27
% d) ser inteligível por si mesmo (dispensar a consulta ao trabalho);
% e) evitar a enumeração de tópicos;
% f) evitar repetição de frases inteiras do trabalho;
% g) respeitar a ordem em que as ideias ou fatos são apresentados;
% h) evitar o uso de parágrafos, frases negativas, abreviaturas, fórmulas,
% diagramas, quadros, equações, etc.;
% i) evitar o uso de símbolos e contrações que não sejam de uso comum;
% j) em trabalhos acadêmicos não se precede o resumo com o uso da referência
% do documento.
% O resumo deve conter:
% a) 150 a 500 palavras para trabalhos acadêmicos: dissertações, teses, TCCs
% e relatórios;
% b) 100 a 250 palavras para os artigos de periódicos.

\begin{center}
  \textbf{RESUMO}  
\end{center}

\noindent
\lipsum[10][1-100]

\vspace{1\baselineskip}
% no mínimo precisam constar três palavras-chave e no máximo cinco.
\noindent
\textbf{Palavras-chave}: palavra\_1, palavra\_2, palavra\_3, palavra\_4, palavra\_5.
\pagebreak


% resumo língua estrangeira

\begin{center}
  \textbf{ABSTRACT}  
\end{center}

\noindent
\lipsum[10][1-100]

\vspace{1\baselineskip}
% no mínimo precisam constar três palavras-chave e no máximo cinco.
\noindent
\textbf{Keywords}: word\_1, word\_2, word\_3, word\_4, word\_5.
\pagebreak


% lista de ilustrações
\cleardoublepage
\phantomsection
\renewcommand{\listfigurename}{\normalfont\large\textbf{LISTA DE ILUSTRAÇÕES}} 
\addcontentsline{toc}{section}{\listfigurename}
\listoffigures
\pagebreak


% lista de tabelas
\renewcommand{\listtablename}{\normalfont\large\textbf{LISTA DE TABELAS}}
\addcontentsline{toc}{section}{\listtablename}
\listoftables
\pagebreak


% lista de símbolos
\renewcommand{\listacronymname}{\hspace*{0.25\textwidth}
\centering\large\textbf{LISTA DE ABREVIATURAS E SIGLAS}}


\printglossary[type=\acronymtype,style=long,title={\listacronymname}]


\noindent
Primeira aparição no texto: \gls{abnt}\\
Segunda aparição no texto: \gls{abnt}
\pagebreak

% lista de símbolos
\renewcommand{\listsymbolname}{\hspace*{0.35\textwidth}
\centering\large\textbf{LISTA DE SÍMBOLOS}}
\printunsrtglossary[type=symbols,style=long,title={\listsymbolname}]
\noindent
Primeira aparição: \gls{laplaciano}\\
Segunda aparição: \gls{laplaciano}
\pagebreak

% sumário
\renewcommand{\contentsname}{\normalfont\large\textbf{SUMÁRIO}} 
\cftsetindents{chapter}{0em}{5em}
\cftsetindents{section}{0em}{5em}
\cftsetindents{subsection}{0em}{5em}
\setcounter{tocdepth}{2}
\tableofcontents
\pagebreak
\restorefortoc

% para referência, tabela com elementos de trabalhos acadêmicos, comente ou delete a linha antes de finalizar
\begin{table}[htbp]
  \centering
  \begin{tabular}{>{\bfseries}l l l c c}
    \toprule
    \textbf{Parte} & \multicolumn{2}{c}{\textbf{Elemento}} & \textbf{Obrigatoriedade} & \textbf{FEITO} \\ 
    \midrule\midrule
    
    \multirow{2}{*}{Parte externa} 
      & \multicolumn{2}{l}{Capa} & Obrigatório & SIM \\ 
      & \multicolumn{2}{l}{Lombada} & Opcional\tablefootnote{Lombada obrigatória apenas para trabalhos encadernados} & --- \\ 
    \midrule
    
    \multirow{21}{*}{Parte interna} 
      & \multirow{13}{*}{Elementos pré-textuais} 
      & Folha de rosto & Obrigatório & SIM \\ 
      & & Errata & Opcional & --- \\ 
      & & Folha de aprovação & Obrigatório\tablefootnote{A folha de aprovação é obrigatória para Teses e Dissertações.} & SIM\\ 
      & & Dedicatória & Opcional & SIM \\ 
      & & Epígrafe & Opcional & SIM \\ 
      & & Resumo na língua vernácula & Obrigatório & SIM \\ 
      & & Resumo em língua estrangeira & Obrigatório & SIM \\ 
      & & Lista de ilustrações & Opcional & SIM \\ 
      & & Lista de tabelas & Opcional & SIM \\ 
      & & Lista de abreviaturas e siglas & Opcional & SIM \\ 
      & & Lista de símbolos & Opcional & SIM \\ 
      & & Sumário & Obrigatório & SIM \\ 
    \cline{2-5}
    
      & \multirow{3}{*}{Elementos textuais} 
      & Introdução & Obrigatório & SIM \\ 
      & & Desenvolvimento & Obrigatório & SIM \\ 
      & & Conclusão & Obrigatório & SIM \\ 
    \cline{2-5}
    
      & \multirow{5}{*}{Elementos pós-textuais} 
      & Referências & Obrigatório & SIM \\ 
      & & Glossário & Opcional & --- \\ 
      & & Apêndice & Opcional & SIM \\ 
      & & Anexo & Opcional & SIM \\ 
      & & Índice & Opcional & --- \\ 
    \bottomrule
  \end{tabular}
  \caption{Elementos já implementados. Aqueles com --- na coluna FEITO não serão implementados no momento ou no futuro próximo.}
  \label{tab:tabela_elementos}
\end{table}
\vfill
\pagebreak

\pagestyle{main} % passa a mostrar os números de páginas
% a partir daqui são os elementos pós-textuais

\chapter{INTRODUÇÃO}
\lipsum[1][1-2000]
\section{SUBTÍTULO}
\lipsum[1][1-2000]
\subsection{Seção ternária}

\begin{center}
    \section*{TÍTULO NÃO NUMERADO CENTRALIZADO}
\end{center}

\chapter{DESENVOLVIMENTO}
\input{Textuais/desenvolvimento}

\chapter{CONCLUSÃO}
\input{Textuais/conclusão}

\chapter*{REFERÊNCIAS}
\input{PosTextuais/referências}

\chapter*{APÊNDICE A}
\input{PosTextuais/apendice-a}

\chapter*{ANEXO 1}
\input{PosTextuais/anexo-1}

\end{document}
