% Elemento obrigatório que contém a apresentação concisa dos pontos
% relevantes do trabalho, fornecendo visão rápida e clara do conteúdo e das conclusões
% do mesmo. Deve ser elaborado conforme os requisitos estipulados pela NBR 6028
% (ABNT, 2021).
% Deverá conter, de forma clara e sintética, a natureza do trabalho, o objetivo, o
% método, os resultados e as conclusões, visando fornecer elementos para o leitor
% decidir sobre a consulta do trabalho no todo. O resumo deve, conforme ABNT (2021):
% a) ser iniciado com frase significativa, explicando o tema principal do
% documento;
% b) ser redigido em parágrafo único em linguagem clara e objetiva, sem recuo
% de parágrafo na primeira linha;
% c) convém o uso dos verbos deve ser na voz ativa, utilizando-se de preferência
% a terceira pessoa do singular;
% 27
% d) ser inteligível por si mesmo (dispensar a consulta ao trabalho);
% e) evitar a enumeração de tópicos;
% f) evitar repetição de frases inteiras do trabalho;
% g) respeitar a ordem em que as ideias ou fatos são apresentados;
% h) evitar o uso de parágrafos, frases negativas, abreviaturas, fórmulas,
% diagramas, quadros, equações, etc.;
% i) evitar o uso de símbolos e contrações que não sejam de uso comum;
% j) em trabalhos acadêmicos não se precede o resumo com o uso da referência
% do documento.
% O resumo deve conter:
% a) 150 a 500 palavras para trabalhos acadêmicos: dissertações, teses, TCCs
% e relatórios;
% b) 100 a 250 palavras para os artigos de periódicos.

\begin{center}
  \textbf{RESUMO}  
\end{center}

\noindent
\lipsum[10][1-100]

\vspace{1\baselineskip}
% no mínimo precisam constar três palavras-chave e no máximo cinco.
\noindent
\textbf{Palavras-chave}: palavra\_1, palavra\_2, palavra\_3, palavra\_4, palavra\_5.
\pagebreak
